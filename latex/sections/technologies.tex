\setauthor{Halilovic Ema}
Bei der Auswahl der Technologien für den serverseitigen Bereich wurden folgenden Kriterien berücksichtigt:
Die Auswahl der Technologien für den serverseitigen Bereich berücksichtigt die Erfüllung folgender Kriterien:
\begin{compactitem}
    \item Sie sollen auf dem neuesten Stand der Technik sein
    \item Ausreichende Dokumentation mit ständiger Weiterentwicklung verfügen
    \item Eine weite Auswahl an \glspl{fw} und Funktionalitäten unterstützen in den Bereichen:
    \begin{compactitem}
        \item REST
        \item Filemanagement
        \item Datenbank
    \end{compactitem}
    % \item Und schon erste Erfahrungen in der dazugehörigen Programmiersprache
\end{compactitem} 
% TODO listin gwas die Frameworks machen können in welchem bereich

Neben der Erfüllung der oben genannten Kriterien wurden bei der Entscheidung zur Verwendung der Technologien auch bereits vorhandene Praxiserfahrungen berücksichtigt. 
Eine Übersicht sowie nähere Beschreibung der für die vorliegende Arbeit verwendeten Technologien sind in den folgenden Unterkapiteln gegeben.
Allerdings wurden die endgültigen Technologien aufgrund schon vorhandener Praxiserfahrung getroffen. 
% Dazu finden sich mehrere \glspl{fw}, welche in Fragen kommen.

\section{PostgreSQL}
\begin{wrapfigure}{r}{0.3\textwidth}
  \begin{center}
      \includegraphics[width=0.2\textwidth]{pics/logos/postgres.png}
      \caption{PostgreSQL Logo}
  \end{center}
\end{wrapfigure}
\emph{PostgreSQL} ist ein Open Source System für relationale Datenbanken. 
Es wurde 1986 erstmals an der University of Califorinia at Berkeley entwickelt, was eine Entwicklung seit 35 Jahren bedeutet.  
\emph{PostgreSQL} läuft auf den gängigsten Betriebssystemen im Terminal.
Durch die Entwicklung des Systems seit rund 35 Jahren, entstanden ebenso mehrere User-Interfaces, wie beispielsweise PgAdmin. 
\cite{PostgreSQLAbout}

PgAdmin ist das am weitesten verbreitete Administrationstool für PostgreSQL. 
\cite{PgAdminAbout}
Es erleichtert die Ausführung von \gls{sql}-Befehlen, sowie den Wechsel verschiedener Tabellen.

Eine gute Dokumentation und schon vorhandene Praxiserfahrung haben diese Datenbank hervorgehoben und zur endgültigen Entscheidung geführt. 
In dieser Arbeit wurde PgAdmin 4 verwendet, um Daten mittels Port-forwarding auf der Cloud zu überprüfen. 
Durch die Visualisierung der Tabellen war diese Aufgabe einfacher als mit dem Terminal allein. 


\section{Quarkus}
Das open-source \gls{fw} Quarkus, wird verwendet um cloud-native Projekte in Java zu entwickeln. 
Vorteile dieses \glspl{fw} sind die kurzen Startzeiten, sowie der geringe Arbeitsspeicherverbrauch.
\cite{QuarkusHomepage}

Nach der Erstellung eines neuen Projekts wird standardmäßig eine \hyperref[ch::MavenTool]{Maven}-Struktur erstellt, sowie Quarkus-Datein, welche die Projekteinstellungen modifizieren können, wie zum Beispiel die \emph{application.properties}-Datei. 
Zusätzlich verfügt Quarkus über eine Vielzahl von Extensions, welche durch Command-line-Befehle oder händisch zu Projekten hinzugefügt werden können. 
Um dies und weitere Quarkus-Aktionen zu vereinfachen, bietet dieses \gls{fw} ein zusätzliches Quarkus-\gls{cli}.
\cite{QuarkusAbout, QuarkusFirstApplication}


\subsection{Maven}
\label{ch::MavenTool}
Da die Kompilierungsprozess eines Projektes recht komplex werden können, wird Maven verwendet, um diese zu vereinfachen.
Ein gutes Beispiel eines komplexen Kompilierungsprozess ist die Ausführung eines Projekts auf unterschiedlichen Geräten mit verschiedener Hardware und Konfigurationen.
%Meist sind lokale Konfigurationen der Auslöser dafür.
Durch die Verwendung von Maven wird garantiert, dass dieses Problem nicht auftritt, da die einzige Vorraussetzung des Zielgerätes nun ist, dass Maven installiert und eingerichtet ist.

Mit einem "Maven-Ordner" können Projekte mit gewohnten Maven-Befehlen ausgeführt werden, ohne dass eine Installation des Tools notwendig ist.
Quarkus-Projekte verwenden von Mavens \emph{pom.xml}-Datei, um zum Beispiel die verwendete Java Version oder alle verwendeten Extensions abzuspeichert.
Zusätzlich ist es möglich, ein einheitliches System für Projektkonfigurationen zu bieten.
Dadurch müssen Einstellungen nicht mehr manuell bei Gerätewechsel getroffen werden. 
\cite{MavenAbout}
%In Quarkus-Projekten werden in der \emph{pom.xml}-Datei zum Beispiel die verwendete Java Version, oder alle verwendeten Extensions gespeichert.


\subsection{\gls*{jdbc} Driver - PostgreSQL}
Für Quarkus Projekte gibt es eine Extension namens \textit{"\gls{jdbc} Driver - PostgreSQL"}.
Diese ermöglicht eine Verbindung zu PostgreSQL-Datenbanken. 
In Java versteht man unter der \gls{jdbc} eine \gls{api} für Java-Anwendungen. 
Die \gls{jdbc}-Driver sind Implementationen der \gls{api} für den benötigten Fall, wie hier für PostgreSQL \cite{StackOFJDBC}.


Um die Extension verwenden zu können und eine Datenbankverbindung aufzubauen, müssen in den \emph{application.properties} einige zusätzlichen Konfigurationen eingefügt werden. 
Wichtig sind Informationen, wie die Art der Datenbank, der Pfad, um diese zu erreichen, und die Login-Daten eines berechtigten Nutzeres \ref{lst:quarkusDatasource}:

\begin{lstlisting}[caption=Beispielkonfigurationen,label=lst:quarkusDatasource]
  quarkus.datasource.db-kind=postgresql 
  quarkus.datasource.username=meinUser
  quarkus.datasource.password=meinPassword
  quarkus.datasource.jdbc.url=jdbc:postgresql://<URL>:<Port>/<meinName>
\end{lstlisting}

\subsection{Hibernate ORM mit Panache}
Hibernate ORM ist der \gls{orm} für Quarkus. 
Da in Java objekt-orientiert programmiert wird und PostgreSQL eine relationale Datenbank ist, muss eine Möglichkeit gefunden werden, wie die in Java erstellten Objekte in die Datenbank übertragen werden. 
Da kommt ein \gls{orm} ins Spiel. 
Dieser "übersetzt" den Java-Code für die Datenbank. 
Dadurch können Java-Klassen als Objekte persistiert werden und gelöscht werden, ohne dass komplexe Codezeilen konstruiert werden müssen. 
\cite{ORMAbout}

Panache bietet zusätzliche Klassen mit Funktionen, von denen abgeleitet werden kann. 
Diese erleichtern das Arbeiten mit angelegten Entitäten, besonders bei \gls{crud}-Operationen.
Zum Beispiel ist für das Persistieren eines neue Objekts lediglich ein Aufruf der \emph{.persist()}-Methode notwendig.
\cite{HibernateORMwithPanache}

\subsection{REST-Easy}
\gls{rest}-Easy ist eine Erweiterung, die es ermöglicht, im Quarkus Projekt mit Jakarta RESTful Web Servies zu arbeiten. 
%\gls{rest} hat sich als Standard für Mikroservice-Anwendungen festgelegt.\cite{RESTAbout}
Das heißt, dass durch diese Extension im Projekt \gls{api}s erstellt werden können. 
Im Code werden diese erstellt durch die zweit Annotationen \emph{@Path} und \emph{@<beliebige HTTP-Methode>}.

Folgende \gls{http}-Methoden werden in dieser Arbeit verwendet:
\begin{compactitem}
    \item GET
    \item POST 
    \item DELETE
\end{compactitem}

\subsection{Swagger-ui}
Swagger UI ist ein Tool, welches beim Testen einer \gls{rest}-\gls{api} hilft. 
Es ist open-source und liefert mehrere Funktionalitäten.
Für diese Arbeit wurde nur ein Teil des Tools verwendet, nämlich die Visualisierung der Schnittstellen. 
Diese bietet, je nach definierter Variable, vorgefertigte Requests, welche mit Mausklick ausgeführt werden können, und jederzeit bearbeitbar sind.
Ebenso zeigt sie die dafür benötigte \gls{http}-Methode. 
\cite{SwaggeruiAbout}

\subsection{JUnit}
JUnit steht für Java Unit und ist ein Test\gls{fw}. 
Es hat sich als der Standart dieser Programmierspraache festgelegt und ist spezialisiert auf die Überprüfung von Methoden und Klassen.
Da es open-source ist, besteht eine ausreichliche Dokumentation.
Durch die Popularität, inspirierte dieses Konzept die Testoptionen für andere Programmiersprachen.

JUnit ist in vielen der gängisten \gls{ide}s inkludiert. 
Zum Anlegen einer Testklasse wird gewöhnlich der Name der zu testenden Klasse genommen mit \emph{.test} als Suffix. 
Die benötigten Repositories werden simuliert, sodass diese wie gewohnt verwendet werden können.
\cite{JUnitAbout}


\section{IntelliJ IDEA}
\begin{wrapfigure}{r}{0.3\textwidth}
  \begin{center}
      \includegraphics[width=0.2\textwidth]{pics/logos/intellij.png}
      \caption{IntelliJ Logo}
  \end{center}
\end{wrapfigure}
IntelliJ IDEA ist eine \gls{ide}, welche von JetBrains entwickelt wurde. 
Diese ist ausgelegt für Java- und Kotlin-Projekte und assistiert bei verschiedensten \gls{jvm}-\glspl{fw}. 
Durch eingebaute Features erleichtert diese Entwicklungsumgebung das Programmieren für Nutzende. 
Plug-Ins ermöglichen es, Datenbankverbindungen und weiteres in der \gls{ide} zu konfigurieren, sodass dem Entwickler oder der Entwicklerin eine Übersicht von benötigten Informationen gegeben werden kann.
\cite{IntelliJIDEA}

IntelliJ ist nicht Open Source und benötigt ein kostenpflichtiges Abonnement. 
Als Schüler und Schülerin kann nach der Angabe von Schulinformationen eine gratis Lizenz für 6 Monate alle JetBrains Produkte beantragt werden. 
Wäre der Benefit nicht vorhanden, hätten wir die \gls{ide} nicht gewählt, sondern Visual Studio Code \ref{ch::vsc} für den ganzen Entwicklungsprozess verwendet. 
\cite{IntelliJIDEAPricing}

\section{Visual Studio Code}
\label{ch::vsc}
\begin{wrapfigure}{r}{0.3\textwidth}
  \begin{center}
      \includegraphics[width=0.2\textwidth]{pics/logos/vscode.jpg}
      \caption{VS Code Logo}
  \end{center}
\end{wrapfigure}
Visual Studio Code ist ein Open Source Code Editor für verschiedene Zwecke. 
Das Programm ist auch unter dem Namen \emph{VS Code} bekannt und wurde von Microsoft lizensiert. 
Der offen verfügbare Quellcode führt dazu, dass die Community weiterhin Erweiterungen entwickelt.
\cite{vscodeAbout}

Der Editor läuft auf den Betriebssystemen Windows, macOS und Linux. 
Bei der Installation wird anfangs nur ein Language Support für JavaScript, TypeScript und Node.js mitgeliefert. 
Durch die vielen Erweiterungen kann sich VS Code individuell angepasst werden. 
Ebenso kann es als \gls{ide} genutzt werden, wenn eine Extension für die gewünschte Sprache gefunden werden kann. 
\cite{vscodeAboutGs}

Der Editor wurde für die letzten Schritte der Arbeit genutzt. 
Dazu zählen die Konfiguration der benötigten Cloud-Dateien, sowie das von Frondend und Backend. 

\section{LeoCloud}
Die LeoCloud ist ein Projekt der HTL Leonding mit Aberger Christian als Projektleiter. 
Sie ermöglicht es mittels Kubernetes, eine beliebige Anwendung auf einer Cloud laufen zu lassen. 
Um diese Cloud nutzen zu können, ist eine E-Mail-Adresse mit der HTL-Leonding-Schulsignatur notwendig. 
Da dies ein schulinternes Projekt ist, war eine Kommunikation mit den Entwicklern gegeben, das heißt, dass Fragen sofort beantwortet werden konnten. 
\cite{LeoCloudAbout}

Die Cloud wurde für die Anforderungen genutzt, dass Projekt für alle Geräte zugänglich zu machen. 
Durch ein Deployment auf die LeoCloud war die Anwendung über den Browser erreichbar und somit ein endgültiges Produkt. 

\subsection{Kubernetes}
\begin{wrapfigure}{r}{0.3\textwidth}
  \begin{center}
    \includegraphics[width=0.2\textwidth]{pics/logos/k8s.png}
   \caption{Kubernetes Logo}
  \end{center}
\end{wrapfigure}
Kubernetes ist eine Anwendung, die für das Containern von Applikationen entwickelt wurde. 
Sie ist Open Source und weit verbreitet. 
Ebenso basiert sie auf den Produktionworkloads von Google.
Alternativ ist Kubernetes unter dem Namen \emph{k8s} bekannt.
\cite{k8sAbout}

Wie vorhin erwähnt, nutzt die LeoCloud Kubernetes, um Anwendungen auf die Cloud zu laden. 
Besonders wird dabei das \gls{cli} verwendet. 

Da das Arbeiten am Terminal unübersichtlich werden kann, wurde ebenso die Extension \emph{Kubernetes von Microsoft} in Visual Studio Code genutzt.
\cite{k8sAboutExtension}

\section{GitHub}
\begin{wrapfigure}{r}{0.3\textwidth}
  \begin{center}
      \includegraphics[width=0.2\textwidth]{pics/logos/github.png}
      \caption{GitHub Logo}
  \end{center}
\end{wrapfigure}
GitHub ist eines der am weitest verbreiteten Entwicklertools auf der Welt. 
Es ist Open Source verfügbar und ermöglicht mit einer Webseite // TODO 
Zusätzlich existieren bereits diverse Apps und Extensions, die die Entwicklung mit GitHub auf anderen Plattformen verbreiten.
\cite{githubAbout}

Durch GitHub Repositories wird das Konzept der \gls{ssot} ermöglicht. 
Bei der Erstellung von Repositories werden mehrere Funktionen geboten, wie Pull Requests. 
Wenn ein Teammitglied eine Änderung des Codes veröffentlicht, überschreiben diese nicht den aktuellen Stand, sondern erzeugen Pull Requests.
Nach einer Überprüfung dank der integrierter Codevorschau können die Requests geschlossen und die neuen Zeilen mit schon vorhandenem Code zusammengeführt werden. 
Alle Änderungen bleiben in Changelogs gespeichert, sodass immer auf alte Versionen zugegriffen werden kann.
\cite{githubAbout}

\section{Notion}
\begin{wrapfigure}{r}{0.3\textwidth}
  \begin{center}
    \includegraphics[width=0.2\textwidth]{pics/logos/notion.png}
   \caption{Notion Logo}
  \end{center}
\end{wrapfigure}

Notion ist eine Art digitales Notizbuch, womit so genannte \emph{Workspaces} erstellt werden können. 
Es ist im Browser, als App in Windows-, MacOS- oder auf iOS- und Android-Geräten verfügbar. 
Workspaces aufgebaut durch mehrere Seiten, welche durch Befehle erstellt werden. 
Diese werden auch direkt miteinander verlinkt.
Seiten lassen sich mit beliebigem Inhalt füllen und durch einfache \emph{/-Befehle} ist es möglich z. B. ein Kanban Board erstellen. 
\cite{NotionAbout}

Notion ist gut für Neueinsteiger, da alle Funktionalitäten gut beschrieben sind und es benutzerfreundlich gestaltet ist.
Workspaces lassen sich mit anderen teilen, wodurch jeder User auf eine \gls{ssot} Version des Workspaces zugreifen kann.
Für die Bearbeitung in echt-zeit wird Internet benötigt, falls dieses jedoch ausfällt, werden die Änderungen gespeichert und synchronisiert, sobald wieder eine Netzwerkverbindung aufgebaut wurde.
\cite{NotionAbout}

Das Feature von geteilten Workspaces wird in dieser Arbeit als Erleichterung der Kommunikation zwischen Teammitgliedern genutzt, um gemeinsame Notizen und Termine festzuhalten. 
Durch die Verwendung dieses Tools wurde sichergestellt, dass jedes Mitglied eine \gls{ssot}-Version der geteilten Informationen hatte.
Für die selbstständige Arbeit wurde das Tool ebenso genutzt, um wichtige Webseiten zu speichern, sowie Zwischennotizen zu erstellen. 

\section{PlantUML}
\begin{wrapfigure}{r}{0.3\textwidth}
  \begin{center}
      \includegraphics[width=0.2\textwidth]{pics/logos/plantuml.png}
      \caption{PlantUML Logo}
  \end{center}
\end{wrapfigure}
PlantUML ist ein Tool, welches die Erstellung von \gls{uml}-Diagrammen vereinfacht. 
Das Ziel von \gls{uml} ist es, Tools zu liefern, um Systeme zu visualisieren.
\cite{UMLPaper}
Dafür gibt es verschiede Arten von \gls{uml}-Diagrammen, wie beispielsweise Objektdiagramme oder Sequenzdiagramme.
\cite{PlantUML}

Für diese Arbeit wird die gratis Extension \emph{PlantUML von jebbs} verwendet. 
Sie bietet eine Preview des Diagramms, sowie Exportmöglichkeiten. 
\cite{PlantumlExtension}
Die Visualisierung unserer Datenstruktur geschieht durch die Erstellung eines \gls{erd}s mit den PlantUML Klassendiagrammen. 
Die Beziehungen zwischen den Entitäten werden hier mittels der Krähenfußnotation dargestellt.

\section{JSON Web Token}
\begin{wrapfigure}{r}{0.3\textwidth}
  \begin{center}
      \includegraphics[width=0.2\textwidth]{pics/logos/jwt.png}
      \caption{JWT Logo}
  \end{center}
\end{wrapfigure}
JSON Web Token, oder \gls{jwt}, ist ein, durch RFC 7519 gekennzeichneter, Standard für die sichere Datenübertragung. 
Die Token übermitteln Daten als digital signierte \gls{json}-Objekte.
Die Signatur erfolgt durch eine Verschlüsselung mittels Secret oder Schlüsselpaar. 
Secrets werden mit einem HMAC-Algorithmus erstellt, wobei Schlüsselpaare RSA oder ECDSA verwenden
\cite{JWTAbout}

In Tokens kann eine freie Anzahl von wählbare Daten, wie selbst definierte Adjektive, gespeichert werden. 
Typischerweise werden Tokenart und Verschlüsselungsalgorithmus im Header des Tokens gespeichert und die individuell definierten in der \emph{Payload}.
Der Token wird bei einem erfolgreichem Login Request erhalten und danach bei \gls{http}-Requests im Authorization Header mitgeschickt. 
Damit der Token nicht unendlich lang genutzt werden kann, ist es wichtig, eine \gls{ttl} festzulegen. 
Bei Interesse an den gespeicherten Informationen kann jeder JWT auf \href{https://jwt.io}{https://jwt.io} enkodiert werden.
\cite{JWTAbout}

\section{Homebrew}
\begin{wrapfigure}{r}{0.3\textwidth}
  \begin{center}
      \includegraphics[width=0.1\textwidth]{pics/logos/brew.png}
      \caption{Homebrew Logo}
  \end{center}
\end{wrapfigure}
Homebrew ist ein Paketmanager für Linux und macOS. 
Er wurde in Ruby von Max Howell programmiert und vereinfacht Installation und Aktualisierung von Tools und Dependencies. 
Dazu müssen sie im Homebrew-Repository enthalten sein. 
Alle Installationen werden im selben Ordner gespeichert, wodurch bei Bedarf des Pfades dieser sofort gefunden wird. 
Um Homebrew zu verwenden, wird im Terminal Stichwort \emph{brew} genutzt. 

Die meisten der oben angeführten Technologien sind im Homebrew-Repository enthalten. 
Durch die Entwicklung auf einem macOS-System, war dieser Paketmanager ein sehr hilfreiches Tool. 
\cite{brewAbout}

%\subsection{Deeper}
%Nicht mehr im Inhaltsverzeichnis.

%\subsubsection{Deepest}
%Vermeide mich.