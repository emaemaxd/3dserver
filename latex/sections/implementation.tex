\section{Design und Planung}

\subsection{Allgemeine Anforderungen}
In den Anfangsgesprächen wurden schon einige Anforderungen für die Planung festgelegt. 
Dabei wurde ein gemeinsamer Workspace in Notion erstellt, sowie ein gemeinsames Github-Repository.
Dadurch wurden für die Termine, sowie den Coding-Prozess eine \gls{ssot} gewährleistet.

Ebenso kamen die ersten Anforderungen der Entitäten zustande für das zukünftige \gls{erd}.

Das Hauptanwendungsszenario der Arbeit wurde definiert als \emph{Ein Nutzer kann Exhibitions erstellen und sich die Exhibitions von anderen Nutzern ansehen.} 
Alle weiteren Diskussionen über die Umsetzung dieses wurden anschließend in einem \gls{erd} zusammengefasst.

%\subsection{Erstellung eines Workspaces in Notion}
\subsection{Erstellung des ERDs}
Um zu gewährleisten, dass die Systemanforderung getroffen werden, ist es notwendig ein entsprechendes \gls{erd} zu erstellen.
Innerhalb eines \gls{erd} werden alle Entitäten, Attribute, sowie deren Beziehungen dargestellt.
Es dient als gemeinsamer Nenner für die Projektentwicklung, da jedes Teammitglied eine eigene Vorstellung der Datenstruktur haben könnte. 
Ein weiterer Vorteil eines \gls{erd}s ist das Potenzial Fehlerquellen zu identifizieren und dadurch zukünfige Problem zu umgehen. 

Nachdem die benötigten Entitäten, sowie deren Attribute festgelegt wurden, müssen nun die Beziehungen zueinander festgelegt werden. 
Dabei wird gewählt zwischen One-to-One, One-to-Many und Many-to-Many Beziehungen, wobei die Many-to-Many Beziehung innerhalb einer Assoziationstabelle dargestellt wird. 

Innerhalb dieser Arbeit ist die Entität \emph{Exhibitions} das zentrale Objekt, was in der Abbildung \ref{fig:erd:lastversion} gut ersichtlich ist.
Die Beziehungen zwischen \emph{Positionen, Exhibits, Rooms und Exhibitions} stellten sich besonders als Herausforderung dar, aufgrund der vielen Abhängigkeiten zueinander. 
Diese wurden besonders erst in der Implementierung ersichtlich, wodurch das \gls{erd} im Laufe der Entwicklung etwas angepasst wurde. 
Weitere Gründe für die Abänderung, waren Planänderungen bei der Kommunikation zwischen zwei Entitäten, sowie der Aufbau der Hauptentitäten \emph{Exhibition und Exhibit}. 
Die Unterschiede zwischen der ersten Version und der finalen sind deutlich (siehe Abb. \ref{fig:erd:firstversion} und \ref{fig:erd:lastversion}).
Besonders lassen sich die oben erläuterten Änderungen erkennen. 
 
\begin{figure}
    \centering
    \includegraphics[scale=0.5]{pics/firstversion_erd.png}
    \caption{Erste Version des ERDs}
    \label{fig:erd:firstversion}
\end{figure}

\begin{figure}
    \centering
    \includegraphics[scale=0.5]{pics/lastversion_erd.png}
    \caption{Finale Version des ERDs}
    \label{fig:erd:lastversion}
\end{figure}


\section{Aufsetzen der Datenbank und des Servers}

Um das Quarkus-Projekt aufzusetzen, gibt es mehrere Möglichkeiten. 
Standardmäßig wird dazu die Webseite \href{https://code.quarkus.io}{code.quarkus.io} verwendet, da hier die gewünschten Abhängigkeiten durch unkompliziertes Auswählen hinzugefügt werden können. 
Für diese Arbeit wurde jedoch das IntelliJ Plug-in \emph{Quarkus} von JetBrains s.r.o. verwendet. 
Dieses bildet dieselben Möglichkeiten, wie die Webseite, in einem eigenem Fenster ab (siehe Abb. \ref{fig:intellij:plugin}).
Zusätzlich werden Tools zur Verfügung gestellt, wie zum Beispiel Code-Assistenz, oder Laufkonfigurationen. \cite{QuarkusPlugin}

\begin{figure}
    \centering
    \includegraphics[scale=0.4]{pics/quarkusplugin.png}
    \caption{Auswählen der gewünschten Quarkus Extensions im Quarkus Plugin}
    \label{fig:intellij:plugin}
\end{figure}

Nach dem Aufsetzen des Projektes, wird eine Datenbankverbindung benötigt. 
Dazu wird PostgreSQL am Gerät installiert und gestarted.
Name, Nutzer und Port der Datenbank können frei gewählt werden, allerdings wurden diese hier nach standardmäßiger Vorgabe gelassen.

Als nächsten Schritt wird eine Verbindung zu diesem lokalen Sysytem aufgebaut. 
Dafür werden die Einstellungen in Abb. \ref{lst:quarkusDatasourceProject} konfiguriert.

\begin{lstlisting}[label=lst:quarkusDatasourceProject]
    quarkus.datasource.db-kind=postgresql
    quarkus.datasource.username = postgres
    quarkus.datasource.password = postgres
    quarkus.datasource.jdbc.url=jdbc:postgresql://localhost:5432/postgres

    quarkus.hibernate-orm.database.generation=drop-and-create
\end{lstlisting}

Falls der Verbindungsaufbau versagt, wirft Quarkus während der Kompilierung Fehlermeldungen. 
Andernfalls können nun die Tabellen angelegt werden. 

\subsection{Umsetzung des ERDs in Quarkus}
Um eine Tabelle zu erstellen wird, wie vorher erwähnt, die Quarkus-Abhängigkeit \emph{Hibernate ORM mit Panache} verwendet.
Mittels dieser wird eine Tabelle erstellt duch die Definition der Annotation \emph{@Entity}.
Üblicherweise benötigt jede Entitäten-Klasse eine definierte Id.
Durch die Ableitung der \emph{PanacheEntity}-Klasse (siehe Abb. \ref{lst:exhibitionEntityCode}, Zeile2) kann jedoch auf diese verzichtet werden, da sie dort schon definiert wurde.

In der Klasse werden alle Attribute als Variablen, gemäß des \gls*{erd}s definiert. 
Die Beziehungen zu anderen Entitäten werden durch die Annotationen \emph{@OneToMany, @ManyToMany und @OnetoOne} über der dazugehörigen Variable dargestellt. 
Auf der gegenseitigen Entität muss ebenso eine Variable erstellt werden, wobei da die Annotation etwas anders aussieht. 
Diese muss die \emph{mappedBy}-Einstellung konfigurieren, da ansonsten die Beziehung mehrmals generiert wird(siehe Abb. \ref{lst:exhibitionEntityCode}, Zeile 4). 
Bei Many-to-Many wird zusätzlich die Annotation @JoinTable hinzugefügt, um die Assoziationstabelle zu konfigurieren (siehe Abb. \ref{lst:exhibitionEntityCode}, Zeilen 5 bis 12). 
Zu den Relationen können zusätzlich noch \gls{cascade} festgelegt werden. 

\begin{lstlisting}[label=lst:exhibitionEntityCode, language=java]
@Entity
public class Exhibition extends PanacheEntity {
    // ...
    @OneToMany(mappedBy = "exhibition", cascade = CascadeType.REMOVE)
    public List<Exhibit> exhibits;

    @ManyToMany
    @JoinTable(
            name = "exhibitions_categories",
            joinColumns = @JoinColumn(name = "exhibition_id"),
            inverseJoinColumns = @JoinColumn(name = "category_id")
    )
    public Set<Category> categories;
}
\end{lstlisting}

Für die Entitäten Theme, Positionen und Rooms hat sich das Team dafür entschieden, die Daten vorgefertigt in die Datenbank zu laden und die dazugehörigen Dateien manuell auf den Server zu speichern. 
Das bedeutet, dass für diese Entitäten keine Methoden angelegt werden müssen, um neue Objekte zu erstellen. 
Da in den \emph{application.properties} die Art der Datenbank auf \emph{drop-and-create} gesetzt wurde, war eine \emph{import.sql}-Datei während der Entwicklung von großer Bedeutung. 
Diese lädt die vordefinierten Daten bei jedem Start automatisch in die Datenbank. 

\subsection{Implementierung der REST-Schnittstellen}

Damit das Frontend auf Daten zugreifen kann, werden Schnittstellen benötigt. 
Diese werden im Backend in den passenden Resource-Klassen erstellt. 
Ein Endpoint wird durch die Verwendung der \emph{@Path}-Annotation erstellt.
Zusätzlich wird definiert, unter welchem Pfad die Methoden erreichbar sind, sowie mittels welcher \gls{http}-Methode darauf zugegriffen werden kann. 

Um die Funktionalität zu gewährleisten, wurde die Extension \emph{Swagger UI} verwendet. 
Diese vereinfacht die Überprüfung des Codes durch eine Visualisierung der Endpoints mit vorgefertigten Abfragen. 
Das Verfahren des \gls{fw}s wird jedoch in Abschnitt \ref{chapter:implementation:tests} näher bearbeitet. 

\subsubsection{Repositories} 

Um in Resource-Klassen die Datenbank anzusprechen, müssen die benötigten Repositories injiziert werden. 
Durch die Verwendung der Panache Extension, implementieren alle Repository-Klassen das \emph{PanacheRepository}, wobei die zugehörige Tabelle in spitzen Klammern angegeben werden muss, wie in Code Beispiel \ref{lst:panacherepo} bei Zeile 2.
Dieses fügt unter anderem einfache Methoden für \gls{crud}-Operationen hinzu, ohne diese zusätzlich definieren zu müssen. 
Dies vereinfacht die Entwicklung enorm, da normalerweise ein Datenbankzugriff in Java viel Boiler-Code 

\begin{lstlisting}[label=lst:panacherepo, language=Java, caption=Teil aus dem Exhibition Repository]
@ApplicationScoped
public class ExhibitionRepo implements PanacheRepository<Exhibition> {
    //...
    public List<ExhibitionWithUserRecord> listAllExhibitionsWithUserField() {
        String sql = "select new 
            org.threeDPortfolioGallery.records.ExhibitionWithUserRecord(e, u.user_name, u.icon_url) from Exhibition e join e.user u left join e.categories c";
        TypedQuery<ExhibitionWithUserRecord> q = getEntityManager()
                .createQuery(sql, ExhibitionWithUserRecord.class);
        var ret = q.getResultList();
        return ret;
    }
}
\end{lstlisting}

Die Repositories ermöglichen ebenso das Definieren von eigenen \gls{jpql}- und \gls{sql}-Befehlen. 
\gls{jpql} ermöglicht, neben der Nutzung der gewohnten \gls{sql}-Keywords, das definieren von Parametern in Queries, sowie die Verwendung von eigenen Records und \gls{dto}s.

\subsubsection{Records und DTOs}
Records sind eine Möglichkeit, einen Datensatz zurückzugeben, mit Werten aus verschiedenen Tabellen. 
Sie können gesetzt werden, jedoch nicht mehr geändert werden. 
Durch ihre kurze schreibweise eignen sie sich besonders gut, Daten aus \gls{sql}-Statements zu speichern und zu returnen.

\gls{dto}s hingegen können mittels Set-Methoden ihre Werte wieder ändern.
Dies ist nicht notwendig beim Auslesen der Datenbank, jedoch benötigt beim Anlegen neuer Objekte. 

\subsubsection{User}
Die Entität eines Benutzers wird User genannt. 
Für den Tabellennamen in der Datenbank wird jedoch eine alternative Bezeichnung definiert.
Dies ist, da der Name \emph{User} in Postgres reserviert ist. 
Wenn dieser Fakt nicht beachtet wird, ist es nicht möglich die Tabelle zu erstellen.
Dies ist der Grund, weshalb im Code \ref{lst:userentity} die Zeile 2 erforderlich ist.

Prinzipiell wird für das Anlegen eines Users in der Datenbank, nur ein Nutzername und ein Passwort benötigt.
Eine Id wird, wie vorhin erwähnt, automatisch generiert. 
Mittels der Attributen der \emph{@Column} Annotation, lassen sich Obligatorische Felder definieren.
Ebenso können Feldlänge, Spaltenname und vieles mehr auf diese Art konfiguriert werden.

Die Passwörter werden nicht als Klartext gespeichert. 
Sie bekommen einen Salt angehängt und werden mithilfe von Base64 verschlüsselt.


\begin{lstlisting}[label=lst:userentity, language=Java, caption=Teil der Entity-Klasse des Users]
@Entity
@Table(name="Users")        
public class User extends PanacheEntity {
    @Column(nullable = false, length = 50)
    public String user_name;
    @Column(nullable = false)
    public String password;
    public String email;

    //...
}
\end{lstlisting}

Der Zugriff auf die Schnittstellen ist gesichert durch ein Tokensystem. 
Nach jedem erfolgreichem Login wird ein Token zurückgesendet, der für die Authentifizierung genutzt werden kann. 
In diesem Token ist der Nutzername gespeichert, die \gls{ttl}, sowie zu welcher Gruppe der Nutzer angehört. 
Diese Tokens lassen sich auf der offiziellen \gls{jwt}-Webseite entschlüsseln, sodass bei Interesse die Informationen jederzeit vom Client abgerufen werden können.

Je nach Gruppe kann auf verschiedene Endpoints zugegriffen werden. 
Die Limitationen werden bei der Definition der Methoden angegeben mittels \emph{@PermitAll}, was keine Limitationen hinzugefügt.
Genauso gibt es jedoch \emph{@RolesAllowed}, was nur definierten Rollen den Zugriff erlaubt.  

\subsubsection{Exhibitions}
Eine der wichtigsten Funktionen der Exhibitions-Resource, ist das Anlegen einer Exhibition. 
Dafür wird eine \emph{@POST}-Methode erstellt, die ein Objekt nach dem \gls{dto} in Codeausschnitt \ref{lst:exhibitiondto} einliest.
\gls{dto}s können auch verschachtelt werden, was in Zeile 12 erkennbar ist. 

In der Klasse wurde ebenfalls die Annotation \emph{@Data} verwendet. 
Diese stammt aus der Java Library Project Lombok und automatisiert die erstellung von Get- und Set-Methoden. 
\cite{Lambok}

\begin{lstlisting}[label=lst:exhibitiondto, language=Java, caption=Exhibition DTO]
    @Data
    public class AddExhibitionDTO {
        String thumbnail_url;
        String title;
        String description;

        Long room_id;

        Long user_id;
        Long[] category_ids;

        AddExhibitDTO[] exhibits;
    }
\end{lstlisting}

Da die Methode ein \gls{dto} erwartet, muss definiert werden, dass etwas eingelesen werden soll. 
In Quarkus werden diese durch \emph{@Consumes} gekennzeichnet. 
Der Codeabschnitt \ref{lst:newExhibitionMethod} Zeile 4 legt fest, dass beim Senden der POST-Methode ein \gls{json}-Objekt erwartet werden soll. 
Der Ausdruck in der Klammer in Zeile 5 im selben Abschnitt beschreibt, wie dieses Objekt konstruiert sein soll.

Beim Aufruf der Methode wird das mitgegebene Objekt in der Variable \emph{newExhibition} gespeichert.
Danach werden zwei leere Variable initialisert. 
Zum einen eine neue Exhibition, zum anderen eine LinkedList bestehend aus Exhibits.
Diese sind getrennt, da alle Exhibits zuerst in die Datenbank eingefügt werden müssen, bevor die Exhibition tatsächlich gespeichert wird. 

Neben den Exhibits, müssen genauso die anderen Werte der mitgegebenen Variable geprüft werden. 
Das heißt, es muss sichergestellt werden, dass keine nicht-existierenden User- oder Category-Ids übergeben wurden. 
Falls etwas einen Fehler während der Überprüfung wirft, Fehlermeldung zurückgegeben mit dem Statuscode 406 "Not Acceptable". 
Gleichzeitig geschieht ein Rollback der davor gespeicherten Objekte. 
Diese Vorgehensweise wird durch die Annotation \emph{@Transactional} definiert. 

\begin{lstlisting}[label=lst:newExhibitionMethod, language=Java, caption=Methode zum Anlegen von Exhibitions]
    @POST
    @Transactional
    @Path("/new")
    @Consumes(MediaType.APPLICATION_JSON)
    public Response postNewExhibition(AddExhibitionDTO newExhibition){
        Exhibition exhibition = new Exhibition();
        List<Exhibit> newExhibitList = new LinkedList<>();
        // ...
    }
\end{lstlisting}

\subsubsection{Filemanagement}
Der File-Upload geschieht während der Bearbeitung der Exhibition im Frontend. 
Nach jeder Dateiauswahl des Nutzers auf der Webseite, wird der Endpoint zuständig für den Fileupload aufgerufen. 
Dieser erwartet sich, anders als die anderen POST-Methoden, kein \gls{json}-Objekt, sondern eine Datei. 
Wiedermals wird dies fesgelegt mittels der \emph{@Consumes}-Annotation, mit \emph{"multipart/form-data"} als Wert . 
Das Limit der akzeptierten Dateien wird in den \emph{application.properties} (Abb. \ref{lst:filesizemax}) festgelegt. 
\begin{lstlisting}[label=lst:filesizemax]
    quarkus.http.limits.max-body-size = 300m
\end{lstlisting}

Diese Methode ist unpraktisch, falls der Nutzer im nachinein hochgeladene Objekte löschen möchte. 
Denn es wurde keine Methode implementiert, die überflüssige Dateien entfernt. 
Da diese Arbeit in einem kleinen Rahmen entwickelt wurde, ohne Absicht für eine Weiterführung nach dem Schulabschluss, war solch eine Rücksicht auf Resourcen nicht nötig.

Für das eigentliche Abspeichern einer Datei, ist die Methode \emph{writeFile()} zuständig (siehe Abb. \ref{lst:fileupload}). 
Diese wird aufgerufen, wenn der Endpoint aufgerufen wird und eine Datei mitgegeben wurde. 
Der erste Parameter der Methode, das Byte-Array, wird mittels eines Inputstreams erstellt. 
Dieser Prozess ist eine Art \gls{boilercode}.
Der Filename wird in einer alternativen Methode erstellt. 
Dafür wird der ursprüngliche Name der Datei genommen und untersucht auf Leerzeichen. 
Sobald diese entfernt wurden, wird der abgeänderte Name zurückgegeben. 

Nachdem das File abgespeichert wurde, wird dessen Pfad mit dem neuen Namen in die Response mitgegeben. 
Diese Antwort wird dann als Wert erwartet als URL, beim Anlegen neuer Exhibits. 

%  POST-Methode der Exhibition aufgerufen wird. 
% Da jedes Exhibit eine Datei beinhaltet muss diese hochgeladen werden, um den Abruf ohne fehlende Daten zu ermöglichen.

\begin{lstlisting}[label=lst:fileupload, language=Java, caption=Hochladen der Dateien]
    private void writeFile(byte[] content, String filename) throws IOException {
        File file = new File(filename);
        if (!file.exists()) {
            file.createNewFile();
        }
        FileOutputStream fos = new FileOutputStream(file);
        fos.write(content);
        fos.flush();
        fos.close();
    }
\end{lstlisting}



\section{Tests}
In diesem Kapitel werden die verschiedenen Testverfahren aufgezählt und etwas erläutert.  

\subsection{Während der Entwicklung}
\label{chapter:implementation:tests}
Für das Testen während der Entwicklung wurde die Extension \emph{swagger-ui} verwendet. 
Erreichbar ist die Testresource unter \emph{http://localhost:8080/q/swagger-ui/}. 
In dem lila Rechteck lässt sich der Pfad erkennen, während sich in dem roten Rechteck der Button zum Ausprobieren befindet. 
Nachdem das "Try it out" geklickt wurde, können die benötigten Werte für den Endpoint eingegeben werden (siehe Abb. \ref{fig:implementation:swaggerui}). 
Der Pfad kann bei GET-Requests einfach in einem Browserfenster eingegeben werden und liefert dieselbe Antwort, wie das Tool. 

Für die Entwicklung an \gls{jwt}-geschützten Endpoints wurde der grün eingerahmte Button aus Abbildung \ref{fig:implementation:swaggerui} benötigt. 
Hier können Token eingegeben werden, wodurch normal getestet werden kann. 
Um diesen \gls{jwt} zu erhalten, wurde ein separater Endpoint zum Testen entwickelt. 

\begin{lstlisting}[label=lst:test:testjwt, language=Java]
@GET
@Produces(MediaType.TEXT_PLAIN)
public Response getJwt(){
    String jwt = jwtService.generateJwt();
    return Response.ok(jwt).build();
}
\end{lstlisting}

\begin{figure}
    \centering
    \includegraphics[scale=0.3]{pics/swaggerui.png}
    \caption{Übersicht der erstellten Schnittstellen}
    \label{fig:implementation:swaggerui}
\end{figure}

In Abbildung \ref{fig:implementation:swaggeruipost} lässt sich erkennen, dass die generierte Abfrage für numerische Werte standardmäßig \emph{0} und alphabetische Sequenzen \emph{string} einsetzt. 
Zweiteres ist kein Problem, da jedoch beim Anlegen jedes neuen Objekts die Id generiert wird, ist es einfacher, die Id aus dem Request zu entfernen. 
Falls jedoch ein bestimmter Wert für dieses Attribut gewünscht ist, muss die Einzigartigkeit der angelegten Entitäten beachtet werden. 
Ansonsten werden neue Objekt womöglich nicht gespeichert.

\begin{figure}
    \centering
    \includegraphics[scale=0.3]{pics/swaggeruipost.png}
    \caption{Automatisch generierte JSON-Anfrage}
    \label{fig:implementation:swaggeruipost}
\end{figure}

\subsection{Nach der Entwicklung}
Tests sind eine gute Möglichkeit, zu gewährleisten, dass die Applikation fehlerfrei ihre Aufgaben erledigt. 
Diese müssen eine breite Auswahl von möglichen Situationen abdecken.

In dieser Arbeit wird mit JUnit 5 und Mockito gearbeitet. 
Mockito ermöglicht es, die Repositories zu rekonstruieren. 
\cite{quarkusMockAbout}
\begin{lstlisting}[label=JUnit 5 Abhängigkeit in pom.xml, language=xml]
<dependency>
    <groupId>io.quarkus</groupId>
    <artifactId>quarkus-junit5-mockito</artifactId>
    <scope>test</scope>
</dependency>
\end{lstlisting}

Tests in Quarkus werden mit \emph{@QuarkusTest} gekennzeichnet. 
Mocks werden durch \emph{@InjectMock} erstellt.
Diese stellen alle Methoden aus den gemockten Repositories zur Verfügung.
Zusätzlich besteht die Möglichkeit zu definieren, dass eine Methode vor jedem Test ausgeführt werden soll mittels \emph{@BeforeEach}. 
Die einzelnen Tests werden zusätzlich mit \emph{@Test} annotiert. 
\cite{quarkusMockAbout}

Durch die Priorisierung des Testvorganges während der Entwicklung, wurden die zusätzlich angelegten Testklassen nur für die wichtigsten Endpoints genutzt. 
Dazu zählt beispielsweise die Entität Exhibition. 
Im unteren Beispiel wird ein Requst auf den Endpoint \emph{api/exhibits/} erstellt, wobei keine Exhibition vorhanden ist. 
Dadurch wird der Statuscode 404 erwartet. 
\begin{lstlisting}[label=Ein Exhibition Test, language=Java]
@Test
public void testGetExhibitByIncorrectId() {
    given()
            .when()
            .pathParam("exhibitionId", 1L)
            .get("/api/exhibitions/{exhibitionId}")
            .then()
            .statusCode(404);
}
\end{lstlisting}


\section{Hosten auf einer Cloud}

Bisher zu diesem Zeitpunkt musste die Applikation immer lokal ausgeführt werden. 
Dafür wurden auf jedem Gerät Java, Maven, PostgreSQL, Angular und der node package manager benötigt.


hei ;3


\section{Continious Integration und Deployment}

Zusätzlich war eine Implementation einer Continious Integration, sowie eines Continious Deployments geplant. 
Die beiden Begriffe lassen sich mit CI/CD abkürzen.
Sie sind für eine Automatisierung zuständig während der Softwareentwicklung. 

\subsection{Continious Integration}

Eine Continious Integration bedeutet, dass die Applikationsänderungen mittels mehrerer Überprüfungen auf ihre Funktionalität getestet wird.
Dies ist besonders praktikabel, wenn im Projekt mehrere Entwickler Änderungen im Code vornehmen. 
Eine CI-Pipeline stellt sicher, dass diese keinen Konflikt verursachen bei der Zusammenführung dieser verschiedener Versionen entsteht. 
Dies geschieht, durch Ausführung des Programms und eine Validierung der Tests. 
\cite{cicdabout}

Solch eine Automatisierung kann durch mehrere Arten verwirklicht werden, wie beispielsweise durch GitHub Actions. 
In jedem GitHub Repository gibt es eine Registerkarte namens \emph{Actions}, diese bietet Vorschläge zur Erstellung solch einer Pipeline (siehe Abb. \ref{fig:implementation:ghactions}). 

\begin{figure}
    \centering
    \includegraphics[scale=0.25]{pics/githubactions.png}
    \caption{GitHub Actions Benutzeroberfläche}
    \label{fig:implementation:ghactions}
\end{figure}

Jede Pipeline, oder auch Prozesskette, sind Dateien, welche in einem Ordner namens \emph{workflows} angelegt werden. 
In diesen Dateien sind Prozesse und Konfigurationen definiert, wie beispielsweise der Name der Pipeline.
Ebenso wird definiert, wann diese ausgeführt werden. 
Im Codeausschnitt \ref{lst:cipipeline} in Zeile 3 bis 7 wurde definiert, dass die nachfolgenden Steps bei jedem Push- und Pull-Request am Main Branch ausgeführt werden. 
Ebenso wird der benötigte Runner definiert in Zeile 11. 
Danach werden die einzelnen Schritte definiert, um die \gls{ci} für die Applikation zu erstellen, zum Beispiel die Installation aller node Module. 
In diesem Fall sind alle Steps Teil von einem einzigen Job namens "build". 

\begin{lstlisting}[label=lst:cipipeline, language=bash, caption=Pipeline einer CI]
name: CI

on:
  push:
    branches: [ main ]
  pull_request:
    branches: [ main ]
  # ...
jobs:
  build:
    runs-on: ubuntu-latest

    steps:
      - uses: actions/checkout@v2

      - name: Install all node modules
        run: npm i 
        working-directory: ./3D-Portfolio-Gallery/
    # ...

\end{lstlisting}

\subsection{Continious Delivery}

Eine \gls{cd} ist sozusagen der ein Upgrade der \gls{ci}. 
Das Ziel dieser ist es, die Applikation für die Produktion bereitzustellen. 
Dies geschieht ebenfalls durch eine Pipeline, welche die Anwendung Schritt für Schritt durchtestet und anschließend auf den Produktionsserver lädt. 
\cite{cicdabout}

In diesem Projekt wäre die Pipeline für die Ausführung der \emph{build.sh}-Datei zuständig gewesen. 
Zusätzlich hätte die Entfernung der Pods definiert werden müssen. 

Während der Entwicklung kam es zu dem Fazit, dass eine Konfiguration einer CI/CD Pipeline als nicht nötig empfunden wurde. 
Dies ist einerseits, da nur eine Person an dem Backend arbeitete und somit keine zweite Validierung der Software nötig war, da diese schon manuell ausgeführt wurden. 
Andererseits hätte es eine große Komplexität bedeutet, da Frontend und Backend auf unterschiedlichen Repositories entwickelt wurden und erst zum Schluss vereint. 
 
