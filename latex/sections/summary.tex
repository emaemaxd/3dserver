In diesem Kapitel wird ein Überblick über das Projekt gegeben, wobei ebenso auf die verschiedenen Herausforderungen, die im Laufe des Projekts aufkamen, eingegangen wird.  

\section{Herausforderungen in Quarkus}

Schon zu Beginn der Entwicklung des Quarkus-Servers kamen Hindernisse zustande. 
Das Gerät zur Entwicklung warf bei jedem Kompilierungsversuch Fehlermeldungen, die Schritt für Schritt gelöst werden mussten. 
Das Problem ließ sich nach einigen Änderungen der installierten Entwicklertools, sowie neuen Installationen von Java und Maven, lösen. 
Jedoch nahm es viel Zeit in Anspruch, und führte dazu, dass serverseitige Anforderungen nicht zeitgemäß erfüllt wurden. 

Die Methode \emph{getExhibitionByCategories} hat sich aufgrund der Many-to-Many-Beziehung als große Herausforderung dargestellt. 
Quarkus-Extensions bieten keine Möglichkeit die gewünschte Aufgabe ressourcensparsam umzusetzen. 
Da für die Applikation keine zukünftigen Pläne bestehen, wurde das Problem mittels Java-Listen gelöst. 

Neben der Entwicklung wurde der Code mittels Javadoc dokumentiert. 
Im Nachhinein lässt sich sagen, dass dies sich als nicht sonderlich nützlich herausgestellt hat, da es zeitaufwendig war.
Von dem Zeitfaktor abgesehen, gab es keine weitere Person, die von dem erstellten Dokument profitiert hätte. 
Dadurch lässt sich behaupten, dass normale Kommentare ausgereicht hätten. 

\section{Herausforderungen im Deployment}

Das Deployment der Anwendung stellte sich als eine sehr große Herausforderung dar. 
Neue Änderungen und Statusupdates der LeoCloud wurden nicht online geteilt, was die Nutzung als ehemaliger Schüler oder ehemalige Schülerin deutlich erschwerte. 

Ebenso stellte sich das Deployment verschiedener Services als unübersichtlicher fest als gedacht. 
Dadurch kam es zu mehreren Caching-Problemen, die nur durch sorgfältige manuelle Löschverfahren behoben werden konnten. 

Beide Probleme konnten durch einen regelmäßigen Kontakt mit Herr Aberger Christian, dem Hauptentwickler der LeoCloud, minimiert werden.

\section{Zielerreichung}

Die zwei Hauptaufgaben wurden erreicht, wobei Meileinsteien nicht rechtzeitig abgeschlossen wurden. 

Das Endergebnis dieser Arbeit bietet dem Frontend eine Datenbank zur Datenverwaltung zusammen mit ersten angelegten Objekten. 
Zusätzlich ist die Webseite auf der LeoCloud unter dem Link \href{https://student.cloud.htl-leonding.ac.at/e.halilovic/home}{https://student.cloud.htl-leonding.ac.at/e.halilovic/home} erreichbar.