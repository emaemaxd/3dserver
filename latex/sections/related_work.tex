Für den serverseitigen Datenupload gibt es folgende Kriterien, welche getroffen werden müssen:
\begin{compactitem}
    \item Auf dem neuesten Stand der aktuellen Technik
    \item Gute Dokumentation mit großer Community
    \item Unterstützung von einer weiten Auswahl an \glspl{fw} und Technologien
    \item Persönliche erste Erfahrungen in der dazugehörigen Programmiersprache
\end{compactitem}

Dazu finden sich mehrere \glspl{fw}, welche in Fragen kommen.

%\section{Node.js}
%Node.js ist ein asynchrones Web-\gls*{fw}. 
%Es ermöglicht die Entwicklung mit JavaScript in einer stand-alone Applikation, obwohl in 
% anstelle des gewohnten Browserfensters.
%Dadurch ist die Anwendung auch auf jedem Betriebssystem nutzbar, welches JavaScript interpretieren kann. 
%Bei jeder Node.js Installation, wird der \textit{node package manager}, oder kurz \textit{npm}, mitinstalliert. 

%Dieser Packetmanager hat sein eigenes \gls{cli}, wodurch alle Abhängigkeiten von externen Paketen durch einen Befehl automatisch heruntergeladen werden. 
%\cite{NodejsEduactive}
%Node.js wurde für Netzwerkaplikationen entwickelt, was untypisch ist für JavaScript-Anwendungen.

%\section{Laravel}

\section{Quarkus}


%\section{Spring}





 Citing \cite{InfH} properly.

Was ist eine \gls{guid}?
Eine \gls{guid} kollidiert nicht gerne.

Kabellose Technologien sind in abgelegenen Gebieten wichtig \cite{APCW2006}.

