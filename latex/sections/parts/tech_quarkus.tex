\section{Quarkus}
\begin{wrapfigure}{r}{0.3\textwidth}
    \begin{center}
      \includegraphics[width=0.2\textwidth]{pics/logos/quarkus.png}
     \caption{Quarkus Logo}
    \end{center}
  \end{wrapfigure}
Das Open Source \gls{fw} \emph{Quarkus} wird für die Entwicklung von cloud-native Projekten in Java verwendet. 
Es zeichnet sich unter anderem durch die kurzen Startzeiten, sowie den geringen Arbeitsspeicherverbrauch aus. 
Ebenso sind Quarkus-Projekte dazu optimiert, in Containerumgebungen, wie beispielsweise \emph{Kubernetes}, ausgeführt zu werden.
\cite{QuarkusHomepage, QuarkusAbout}

Ein neues Projekt wird entweder durch das Quarkus-\glsfirst{cli} erstellt oder nach individueller Konfiguration auf \href{https://code.quarkus.io}{code.quarkus.io} heruntergeladen. 
Standardmäßig sind in dem Projektordner \hyperref[ch::MavenTool]{Maven}-Verzeichnis, sowie Quarkus-Dateien enthalten, welche Projekteinstellungen enthalten. 
Eine dieser Konfigurationsdateien wäre zum Beispiel die \emph{application.properties}-Datei. 
\cite{QuarkusAbout, QuarkusFirstApplication}

Zusätzlich verfügt Quarkus über eine Vielzahl von Extensions, welche die gängigsten Technologien unterstützen, wie zum Beispiel einen JDBC-Treiber (siehe \ref{ch::jdbctreiber}) oder REST-Easy (siehe \ref{ch::resteasy}). 
Sie werden durch das \gls{cli} oder manuell hinzugefügt. 
\cite{QuarkusAbout, QuarkusFirstApplication}

Mit Quarkus wurde der Backend-Teil der Arbeit erstellt. 
Die Möglichkeit das Projekt mit zusätzlichen Erweiterungen auszustatten war 

\subsection{Maven}
\label{ch::MavenTool}
\begin{wrapfigure}{r}{0.3\textwidth}
    \begin{center}
        \includegraphics[width=0.1\textwidth]{pics/logos/asf.jpg}
        \caption{\emph{Apache Maven} feather Logo}
    \end{center}
\end{wrapfigure}

\emph{Maven} ist ein Open Source Build-Tool, welches von der Apache Software Foundation lizensiert ist.
Es wurde in Java geschrieben und wird meist in Java-Projekten eingesetzt. 
Durch zentrale Nutzung des \glsentryfirst{pom} werden projektrelevante Informationen gesichert, wodurch der Kompilierungsprozess vereinfacht wird.
\cite{MavenAbout, MavenAbout2gls}

\emph{Maven} sorgt dafür, dass unterschiedliche Geräte eine Applikation ausführen können, ohne manuelle Konfiguration. 
Die einzige Voraussetzung  des Zielgerätes ist, dass \emph{Maven} installiert und eingerichtet ist. 
Ebenso können \emph{Maven-Ordner} gedownloadet werden. 
Diese ermöglichen die Nutzung der Maven Befehle ohne Installation des Tools am Gerät.
\cite{MavenAbout, MavenAbout2gls}

Quarkus-Projekte verwenden \emph{Mavens} \emph{pom.xml}-Datei, um unter anderem die verwendete Java Version oder die verwendeten Extensions abzuspeichern.
Zusätzlich wird ein einheitliches System für Projektkonfigurationen geboten. 
Dadurch müssen, wie oben erwähnt, Einstellungen bei Gerätewechsel nicht mehr manuell geändert werden. 
\cite{MavenAbout}

\subsection{\glsfirst{jdbc} Treiber}
\label{ch::jdbctreiber}
In Java wird für Datenbankverbindungen die \gls{jdbc} \gls{api} verwendet. 
Genauso benötigen Quarkus-Projekte einen datenbankspezifischen \gls{jdbc}-Treiber für einen Zugriff auf die Datenbank. 
In diesem Projekt wurde PostgreSQL als Datenbank gewählt, weshalb die Extension \emph{\gls{jdbc} Driver - PostgreSQL} fundamental für die Entwicklung war. 
\cite{datasourcesAbout}.

Um die Extension zu verwenden und eine Datenbankverbindung aufzubauen, müssen in den \emph{application.properties} die zusätzliche Konfigurationen aus Codeblock \ref{lst:quarkusDatasource} eingefügt werden. 
Wichtig sind Informationen, wie die Art der Datenbank, der Pfad, um diese zu erreichen, und die Login-Daten eines berechtigten Users.
Dazu wird der \gls{boilercode} aus Codeblock \ref{lst:quarkusDatasource} verwendet, wobei die Klammern durch tatsächliche Werte ersetzt werden. 
\cite{datasourcesAbout}

\begin{lstlisting}[caption=Beispielkonfigurationen,label=lst:quarkusDatasource]
  quarkus.datasource.db-kind=postgresql 
  quarkus.datasource.username=<meinUser>
  quarkus.datasource.password=<meinPassword>
  quarkus.datasource.jdbc.url=jdbc:postgresql://<URL>:<Port>/<meinName>
\end{lstlisting}

\subsection{Hibernate ORM mit Panache} 
Da Java eine objekt-orientiert Programmiersprache ist, und PostgreSQL eine relationale Datenbank, können standardmäßig keine Entitäten aus Java in die Datenbank übertragen werden. 
Diese Aufgabe übernimmt ein \gls{orm}, der den Java-Code für PostgreSQL umwandelt. 
Dadurch können Java-Klassen als Objekte persistiert und gelöscht werden, ohne dass komplexer Code konstruiert werden muss. 
Hibernate ORM ist die dafür zuständige Extension für Quarkus. 
\cite{ORMAbout}

Panache bietet zusätzliche Klassen mit Funktionen, von denen abgeleitet werden kann. 
Diese erleichtern das Arbeiten mit angelegten Entitäten, besonders bei \gls{crud}-Operationen.
Zum Beispiel ist für die Persistenz eines neuen Objekts lediglich ein Aufruf der \emph{.persist()}-Methode notwendig, wodurch automatisch eine Entität in die Datenbank eingefügt wird. 
\cite{HibernateORMwithPanache}

\subsection{REST-Easy}
\label{ch::resteasy}
\begin{wrapfigure}{r}{0.3\textwidth}
    \begin{center}
        \includegraphics[width=0.2\textwidth]{pics/logos/resteasy.png}
        \caption{REST Easy Logo}
    \end{center}
\end{wrapfigure}
\gls{rest}-Easy ist eine Quarkuserweiterung, die es ermöglicht, mit RESTful Web Servies zu arbeiten. 
Sie implementiert die Jakarta RESTful Web Services der Eclipse Foundation, sowie die MicroProfile \gls{rest} Client Spezifikation \gls{api}.
\cite{ResteasyAbout}

%\gls{rest} hat sich als Standard für Mikroservice-Anwendungen festgelegt.\cite{RESTAbout}
Dies bedeutet, dass durch diese Extension im Projekt \gls{api}s erstellt werden können. 
Ein Endpoint wird durch Definition zweier Annotationen \emph{@Path} und \emph{@<beliebige HTTP-Methode>} erzeugt.
\cite{ResteasyExtensionAbout}

Folgende \gls{http}-Methoden werden in dieser Arbeit verwendet:
\begin{compactitem}
    \item \textbf{GET}: GET-Requests liefern Daten \cite{httpAbout}
    \item \textbf{POST}: POST-Requests übermitteln Daten wie Entitäten \cite{httpAbout}
    \item \textbf{DELETE}: DELETE-Requests löschen Daten \cite{httpAbout}
\end{compactitem}

Die Beschreibungen der oben angeführten Methoden sind Normen in der Entwicklung mit \gls{http}. 
Eventuell hätte noch \emph{PUT} verwendet werden können.
Diese Methode wird ähnlich wie das \emph{POST} verwendet, jedoch ist es ausgelegt für einzelne Änderungen einer Entität. 
Dadurch hatte sie in dieser Arbeit keinen spezifischen Nutzen.  
\cite{httpAbout}

\subsection{Swagger-ui}
\begin{wrapfigure}{r}{0.3\textwidth}
    \begin{center}
        \includegraphics[width=0.2\textwidth]{pics/logos/swagger.png}
        \caption{Swagger Logo}
    \end{center}
\end{wrapfigure}
Swagger UI ist ein Tool zum Testen von \gls{rest}-\gls{api}s. 
Es ist Open Source und in dieser Arbeit hauptsächlich die Visualisierung der Schnittstellen verwendet. 
Dieser Teil bietet, je nach definierten Parametern, vorgefertigte Requests, welche noch bearbeitet werden können. 
Ebenso zeigen sie die benötigte \gls{http}-Methode und ermöglichen es, den Endpoint mit Mausklick zu testen. 
Dies erspart sehr viel Arbeit während der \gls{api}-Entwicklung.
\cite{SwaggeruiAbout}

\subsection{JUnit 5}
\begin{wrapfigure}{r}{0.3\textwidth}
    \begin{center}
        \includegraphics[width=0.2\textwidth]{pics/logos/junit5-logo.png}
        \caption{JUnit 5 Logo}
    \end{center}
\end{wrapfigure}
\emph{JUnit} steht für Java Unit und ist ein Open Source \gls{fw}. 
Es hat sich als der Standard zum Testen von \emph{Java}-Programmen bewiesen und ist spezialisiert auf die Überprüfung von Methoden und Klassen.
Durch die Popularität besteht eine ausführliche Dokumentation und weites Funktionsspektrum.
Ebenso inspirierte dieses Konzept die \gls{fw}s anderer Programmiersprachen im Testing-Bereich.

JUnit ist in vielen der gängigsten \gls{ide}s inkludiert. 
Zum Anlegen einer Testklasse wird gewöhnlich der Name der zu testenden Klasse verwendet mit \emph{.test} als Suffix. 
Die benötigten Repositories werden simuliert, sodass Methoden aus diesem wie gewohnt verwendet werden können.
\cite{JUnitAbout}
