
\section{Quarkus}
Das open-source \gls{fw} Quarkus, wird verwendet um cloud-native Projekte in Java zu entwickeln. 
Vorteile dieses \glspl{fw} sind die kurzen Startzeiten, sowie der geringe Arbeitsspeicherverbrauch.
\cite{QuarkusHomepage}

Nach der Erstellung eines neuen Projekts wird standardmäßig eine \hyperref[ch::MavenTool]{Maven}-Struktur erstellt, sowie Quarkus-Datein, welche die Projekteinstellungen modifizieren können, wie zum Beispiel die \emph{application.properties}-Datei. 
Zusätzlich verfügt Quarkus über eine Vielzahl von Extensions, welche durch Command-line-Befehle oder händisch zu Projekten hinzugefügt werden können. 
Um dies und weitere Quarkus-Aktionen zu vereinfachen, bietet dieses \gls{fw} ein zusätzliches Quarkus-\gls{cli}.
\cite{QuarkusAbout, QuarkusFirstApplication}


\subsection{Maven}
\label{ch::MavenTool}
Da die Kompilierungsprozess eines Projektes recht komplex werden können, wird Maven verwendet, um diese zu vereinfachen.
Ein gutes Beispiel eines komplexen Kompilierungsprozess ist die Ausführung eines Projekts auf unterschiedlichen Geräten mit verschiedener Hardware und Konfigurationen.
%Meist sind lokale Konfigurationen der Auslöser dafür.
Durch die Verwendung von Maven wird garantiert, dass dieses Problem nicht auftritt, da die einzige Vorraussetzung des Zielgerätes nun ist, dass Maven installiert und eingerichtet ist.

Mit einem "Maven-Ordner" können Projekte mit gewohnten Maven-Befehlen ausgeführt werden, ohne dass eine Installation des Tools notwendig ist.
Quarkus-Projekte verwenden von Mavens \emph{pom.xml}-Datei, um zum Beispiel die verwendete Java Version oder alle verwendeten Extensions abzuspeichert.
Zusätzlich ist es möglich, ein einheitliches System für Projektkonfigurationen zu bieten.
Dadurch müssen Einstellungen nicht mehr manuell bei Gerätewechsel getroffen werden. 
\cite{MavenAbout}
%In Quarkus-Projekten werden in der \emph{pom.xml}-Datei zum Beispiel die verwendete Java Version, oder alle verwendeten Extensions gespeichert.


\subsection{\gls*{jdbc} Driver - PostgreSQL}
Für Quarkus Projekte gibt es eine Extension namens \textit{"\gls{jdbc} Driver - PostgreSQL"}.
Diese ermöglicht eine Verbindung zu PostgreSQL-Datenbanken. 
In Java versteht man unter der \gls{jdbc} eine \gls{api} für Java-Anwendungen. 
Die \gls{jdbc}-Driver sind Implementationen der \gls{api} für den benötigten Fall, wie hier für PostgreSQL \cite{StackOFJDBC}.


Um die Extension verwenden zu können und eine Datenbankverbindung aufzubauen, müssen in den \emph{application.properties} einige zusätzlichen Konfigurationen eingefügt werden. 
Wichtig sind Informationen, wie die Art der Datenbank, der Pfad, um diese zu erreichen, und die Login-Daten eines berechtigten Nutzeres \ref{lst:quarkusDatasource}:

\begin{lstlisting}[caption=Beispielkonfigurationen,label=lst:quarkusDatasource]
  quarkus.datasource.db-kind=postgresql 
  quarkus.datasource.username=meinUser
  quarkus.datasource.password=meinPassword
  quarkus.datasource.jdbc.url=jdbc:postgresql://<URL>:<Port>/<meinName>
\end{lstlisting}

\subsection{Hibernate ORM mit Panache}
Hibernate ORM ist der \gls{orm} für Quarkus. 
Da in Java objekt-orientiert programmiert wird und PostgreSQL eine relationale Datenbank ist, muss eine Möglichkeit gefunden werden, wie die in Java erstellten Objekte in die Datenbank übertragen werden. 
Da kommt ein \gls{orm} ins Spiel. 
Dieser "übersetzt" den Java-Code für die Datenbank. 
Dadurch können Java-Klassen als Objekte persistiert werden und gelöscht werden, ohne dass komplexe Codezeilen konstruiert werden müssen. 
\cite{ORMAbout}

Panache bietet zusätzliche Klassen mit Funktionen, von denen abgeleitet werden kann. 
Diese erleichtern das Arbeiten mit angelegten Entitäten, besonders bei \gls{crud}-Operationen.
Zum Beispiel ist für das Persistieren eines neue Objekts lediglich ein Aufruf der \emph{.persist()}-Methode notwendig.
\cite{HibernateORMwithPanache}

\subsection{REST-Easy}
\gls{rest}-Easy ist eine Erweiterung, die es ermöglicht, im Quarkus Projekt mit Jakarta RESTful Web Servies zu arbeiten. 
%\gls{rest} hat sich als Standard für Mikroservice-Anwendungen festgelegt.\cite{RESTAbout}
Das heißt, dass durch diese Extension im Projekt \gls{api}s erstellt werden können. 
Im Code werden diese erstellt durch die zweit Annotationen \emph{@Path} und \emph{@<beliebige HTTP-Methode>}.

Folgende \gls{http}-Methoden werden in dieser Arbeit verwendet:
\begin{compactitem}
    \item GET
    \item POST 
    \item DELETE
\end{compactitem}

\subsection{Swagger-ui}
Swagger UI ist ein Tool, welches beim Testen einer \gls{rest}-\gls{api} hilft. 
Es ist open-source und liefert mehrere Funktionalitäten.
Für diese Arbeit wurde nur ein Teil des Tools verwendet, nämlich die Visualisierung der Schnittstellen. 
Diese bietet, je nach definierter Variable, vorgefertigte Requests, welche mit Mausklick ausgeführt werden können, und jederzeit bearbeitbar sind.
Ebenso zeigt sie die dafür benötigte \gls{http}-Methode. 
\cite{SwaggeruiAbout}

\subsection{JUnit}
JUnit steht für Java Unit und ist ein Test\gls{fw}. 
Es hat sich als der Standart dieser Programmierspraache festgelegt und ist spezialisiert auf die Überprüfung von Methoden und Klassen.
Da es open-source ist, besteht eine ausreichliche Dokumentation.
Durch die Popularität, inspirierte dieses Konzept die Testoptionen für andere Programmiersprachen.

JUnit ist in vielen der gängisten \gls{ide}s inkludiert. 
Zum Anlegen einer Testklasse wird gewöhnlich der Name der zu testenden Klasse genommen mit \emph{.test} als Suffix. 
Die benötigten Repositories werden simuliert, sodass diese wie gewohnt verwendet werden können.
\cite{JUnitAbout}
