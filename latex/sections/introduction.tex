\setauthor{Ema Halilovic}
\section{Ausgangssituation}
Als Ausgangsbasis wurde die Diplomarbeit '3D Portfolio Gallery' hergenommen und mit folgenden Funktionalitäten erweitert:

\begin{compactitem}
    \item Hosten der Webseite innerhalb einer Cloud
    \item Implementierung von REST-Schnittstellen 
\end{compactitem}

Die genannte Arbeit, erstellt von Litzlbauer Lorenz und Maar Fabian 
ermöglicht die Erstellung von interaktiven und dreidimensionalen Räumen.
Diese Räume werden mittels Angular in einem Browserfenster abgebildet.


Die Bereitstellung der benötigten Daten in Form eines Backends, ist die Hauptbeschäftigung dieser Diplomarbeit.

Dazu wird sich die Frage gestellt \textit{'Welche Arten gibt es, um Daten hochzuladen, sodass sie wieder abgerufen werden können?'}.


\section{Zielsetzung}

Das Hauptziel dieser Arbeit ist es, eine \Gls{api} zu erstellen, welche die Kommunikation zu serverseitigen Daten ermöglicht. 
Dafür ist eine Abbildung der Datenstruktur in einer Datenbank erforderlich. 
Damit diese fehlerfrei erfolgt, wird ein \Gls{erd} verwendet. 