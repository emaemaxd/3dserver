\setauthor{Ema Halilovic}
%Die Einleitung dieser Arbeit beschäftigt sich mit den allgemeinen Informationen und Vorraussetzung. 
%Ebenso werden die Ziele definiert. 

Die genannte Arbeit, erstellt von Litzlbauer Lorenz und Maar Fabian, 
ermöglicht die Erstellung von interaktiven und dreidimensionalen Räumen.
Diese Räume werden mittels Angular in einem Browserfenster abgebildet.

%\section{Ausgangssituation}
Als Ausgangsbasis wurde die Diplomarbeit \emph{3D Portfolio Gallery} herangezogen und mit folgenden Funktionalitäten erweitert:

\begin{compactitem}
    \item Implementierung von \gls{rest}-Schnittstellen 
    \item Hosten der Webseite innerhalb einer Cloud
\end{compactitem}

Die Bereitstellung der benötigten Daten in Form eines Backends, ist die Hauptbeschäftigung dieser Diplomarbeit. 


\section{Zielsetzung}

Das Hauptziel dieser Arbeit ist es, eine \Gls{api} zu erstellen, welche die Kommunikation zu serverseitigen Daten ermöglicht. 
Dafür ist eine Abbildung der Datenstruktur in einer Datenbank erforderlich. 
Damit diese fehlerfrei erfolgt, wird ein \Gls{erd} verwendet. 