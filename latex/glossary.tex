
\newacronym{guid}{API}{Application Programming Interface}
\newacronym{uri}{URI}{Uniform Resource Identifier}
%\newacronym{ssot}{SSOT}{Single Source of Trutz}
\newacronym{cli}{CLI}{Command Line Interface}
\newacronym{jvm}{JVM}{Java Virtual Machine}
\newacronym{jdbc}{JDBC}{Java Database Connectivity}
\newacronym{orm}{ORM}{Object Relational Mapper}
\newacronym{ide}{IDE}{Integrated Development Environment}
\newacronym{uml}{UML}{Unified Modeling Language}
\newacronym{crud}{CRUD}{Create Replace Update Delete}
\newacronym{http}{HTTP}{Hypertext Transfer Protocol}
\newacronym{rest}{REST}{Representational State Transfer}
\newacronym{sql}{SQL}{Structured Query Language}
\newacronym{jpql}{JPQL}{Jakarta Persistance Query Language}
\newacronym{dto}{DTO}{Data Transfer Object}
\newacronym{json}{JSON}{JavaScript Object Notation}
\newacronym{jwt}{JWT}{JSON Web Token}
%\newacronym{ttl}{TTL}{Time To Live}
\newacronym{ci}{CI}{Continious Integration}
\newacronym{cd}{CD}{Continious Deployment}
\newacronym{ghcr}{GHCR}{GitHub Container Registry}

\newglossaryentry{ssot}{
    name=SSOT,
    description={
        Der Begriff steht für \emph{Single Source of Truth}. 
        Er beschreibt ein Konzept, dass jedes Mitglied dieselben Daten besitzt, sodass Lösungsideen immer aktuell sind.  
        \cite{ssotAbout}
    }
}

\newglossaryentry{pom}{
    name=POM,
    description={
        Der Begriff steht für \emph{Project Object Model}. 
        Es ist das Kernstück eines Mavens Projekts, da darin wichtige Informationen stehen. 
        Diese Informationen sind unter anderem Abhängigkeiten, Konfigurationseinstellungen,
        sowie allgemeine Informationen.
        \cite{MavenAbout2gls}
    }
}

\newglossaryentry{ttl}{
    name=TTL,
    description={
        Der Begriff steht für \emph{Time To Live}. 
        Der Wert dessen legt fest, wie lange bestimmte Daten am Server gespeichert werden.
        \cite{ttlAbout}
    }
}

\newglossaryentry{api}{
    name=API,
    description={
        Der Begriff steht für \emph{Application Programming Interface}. 
        Durch APIs können verschieden Applikationen miteinander kommunizieren, ohne die jeweils andere Implementierung wissen zum müssen. 
        \cite{API2022}
    }
}

\newglossaryentry{cascade}{
    name=Cascade Types,
    description={
        Es gibt verschiedene Arten von Cascade Types. 
        Sie definieren, was mit Abhängigkeiten passiert, wenn diese bearbeitet oder gelöscht werden. 
        \cite{CascadeTypes}
    }
}

\newglossaryentry{boilercode}{
    name=Boilerplate Code,
    description={
        Boilerplate Code sind meist Codeblöcke, die an verschiedene Situationen anwenden kann. 
        Meist werden diese nicht geändert, manchmal jedoch müssen Anpassungen gemacht werden.  
        \cite{BoilerCodeAbout}
    }
}


\newglossaryentry{erd}{
    name=ERD,
    description={
        Der Begriff steht für \emph{Entity Relationship Diagram}. 
        Diese Diagramme visualisieren, welche Beziehungen verschiedenen \textit{Entitäten} zueinander haben.  
        \cite{ERD2023}
    }
}

\newglossaryentry{fw}{
    name=Framework,
    plural=Frameworks,
    description={
        Frameworks bieten Funktionen an, ohne dass diese selbst implementiert werden müssen. 
        Sie erleichtern das Programmieren innorm. 
        \cite{FrameworkAbout}
    }
}


% Usage:
% \gls{label} lowercase in text
% \Gls{label} Uppercase in text
% \newacronym{label}{abbrev}{full}
% \newglossaryentry{label}{settings}